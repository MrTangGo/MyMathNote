\section{预备知识}

\subsection{计数法则}

【加法法则】做一件事,完成它可以有$n$类方法,在第一类方法中有$m_{1}$种不同方法,在第二类方法中有$m_{2}$种不同方法,……,在第$n$类方法中有$m_{n}$种不同方法,那么完成这件事共有$m_{1}+m_{2}+\cdots+m_{n}$ 种不同的方法。
\
\begin{itemize}
\item 从武汉到上海有乘火车、飞机、轮船3种交通方式可供选择,而火车、飞机、轮船分别有k1,k2,k3个班次,那么从武汉到上海共有 k1+k2+k3种方式可以到达。sdd 
\item 书架上有数学类书5本,计算机类书3,任取其中一本,共有5+3=8种取法。
\end{itemize}

【乘法法则】做一件事,完成它需要分成$n$个步骤,做第一步有 $m_{1}$ 种不同方法,做第二步有 $m_{2}$种不同方法,……,做第$n$步有$m_{n}$种不同方法,那么完成这件事共有$m_{1} \times m_{2} \times  \cdots \times m_{n}$ 种不同的方法。

\begin{itemize}
\item 从A到B有三条道路,从B到C有两条道路,则从A经B到C有几条道路?共有3×2=6条。
\item 某种样式的运动服的着色由底色和装饰条纹的颜色配成。底色可选红、蓝、橙、黄,条纹色可选黑白,则共有4×2=8种着色方案。

\item 对于7位车牌号,如果要求前3位必须是字母,后4位必须是数字,那么一共有多少种不同的7位车牌号?答:26×26×26×10×10×10×10=175760000种不同的号码。

\item 5封信投入3个不同的信箱,可以有几个方法?答:第一封信有3个方法,第二封信有3个方法,。。。一共 $3^{5}$种方法。

\end{itemize}


\subsection{排列}
随意排列字母a,b,c,通过直接列举,可知一共有6种:

\[
abe,acb,bac,bea,cab,cba 
\]

每一种都称为一 个\textit{ 排列( permutation)} 。因此,3个元素一共有6种可能的排列方式,这个结果能通过乘法法则得到:

在排列中第一个
位置可供选择的元素有3个,第二个位置可供选择的元素是剩下的两个元素之一,第三个
位置只能选择剩下的1个元素。因此,一共有3×2×1=6种可能的排列。

假设有n个元素,那么用上述类似的推理,可知一共有

\[
n×(n-1)×(n-2)×…×3×2×1=n!
\]


种不同的排列方式。\par\vspace{\baselineskip}


【例题】某概率论班共有6名男生、4名女生,有次测验是根据他们的表现来排名次,假设没有两个学生成绩一样。如果限定男生和女生分开排名次,那么一共有多少种可能的名次?

解:男生一起排名次有6!种可能,女生一起排名次有4!种可能,一共有6!×4!=720×24=17280种可能的名次。\par\vspace{\baselineskip}

【例题】Jones女士要把10本书放到书架上,其中有4本数学书、3本化学书、2本历史书和1本语文书。现在 Jones女士想整理她的书,如果相同学科的所有图书都必须放在一起,那么一共可能有多少种放法?

解:数学书的摆放有4!可能。化学书有3!可能。历史书2!种可能。语文书1!种可能。每种书又要放在一起,可以抽象理解为书架上面只有4本书,这4本书的顺序可以是4!种顺序。在一种顺序下的可能性是4!× 3! × 2! × 1! 那么 4!种顺序就是 4!× 4!× 3! × 2! × 1! = 6912 种可能性。\par\vspace{\baselineskip}

【例题】用4个字母进行排列,P、P、P、E,一共有多少种排列的方式?

解:如果上面的三个P都是不同的P,即$P_{1} 、P_{2}、P_{3}、E$那么可以得到的结果是4!种不同的排列。可是考虑其中某一个排列,比如说是EPPP,如果将三个P重新排列:

$$
\begin{array}{cccc}
EP_{1}P_{2}P_{3} &EP_{1}P_{3}P_{2} \\
EP_{2}P_{1}P_{3}  &EP_{2}P_{3}P_{1}  \\
EP_{3}P_{1}P_{2}  &EP_{3}P_{2}P_{1} 
\end{array}
$$


可以看到上面6种排列,实际上都应该归类为1种排列。因此一共有4!/3! = 4种排列。

换一个思路可以这样去理解,把上面的4个字母的想象成4个盒子,如果是先放E的话就有4个选择。

$$
E\square\square\square \hspace{1em} 
\square E\square\square \hspace{1em} 
\square\square E\square \hspace{1em} 
\square\square\square E
$$


剩下的三个盒子,可以放P。如果是不同的P,那么就会有3!种不同的方式去把P放进去。比如说$E\square\square\square$ 这种情况下,就会出现上面的罗列的6种不同放P的方案。

但是如果考虑到是相同的P,所以在放下E的时候,已经决定了剩下P的位置了,因为所有的方式也就只有原来的4种可能了。\par\vspace{\baselineskip}


【例题】用6个字母进行排列,P、P、P、R、R、E,一共有多少种排列的方式?

解:如果上面的每个字母都是独特的,可以等到的结果是6!种不同的排列方式。考虑其中任何一种排列,比如说 EPPPRR。如果分别将3个字母P与两个字母R重新排列,会得到以下的结果:

$$
\begin{array}{cccc}
EP_{1}P_{2}P_{3}R_{1}R_{2}  &EP_{1}P_{2}P_{3}R_{2}R_{1}  \\
EP_{1}P_{3}P_{2}R_{1}R_{2}  &EP_{1}P_{3}P_{2}R_{2}R_{1}  \\
EP_{2}P_{1}P_{3}R_{1}R_{2}  &EP_{2}P_{1}P_{3}R_{2}R_{1}  \\
EP_{2}P_{3}P_{1}R_{1}R_{2}  &EP_{2}P_{3}P_{1}R_{2}R_{1}  \\
EP_{3}P_{1}P_{2}R_{1}R_{2}  &EP_{3}P_{1}P_{2}R_{2}R_{1}  \\
EP_{3}P_{2}P_{1}R_{1}R_{2}  &EP_{3}P_{2}P_{1}R_{2}R_{1}  \\
\end{array}
$$

共计3!× 2!种排列。这一切的排列都是同一个形式EPPPRR。因此,一共有6!/(3!× 2!)种排列的可能。\par\vspace{\baselineskip}


一般来说,对于n个元素,如果其中$n_{1}$个元素彼此相同,另$n_{2}$个彼此相同,$\cdots$,$n_{r}$个也彼此相同,那么一共有

\[
	\frac{n!}{n_{1}!n_{2}! \cdots n_{r}!}
\]
种不同的排列方式。

综合例题:

【1】用红、黄、蓝三种颜色旗子各3面,每次升旗可以选择升一面、两面或三面在某一旗杆上纵向排列,则共可以组成多少种不同的信号?3+3*3+3*3*3 


\subsection{组合}
【例】从$n$ 个元素中随机的找$r$个元素组成一组,例如从字母$A,B,C,D,E$ 中选出二个字母组成一组,会有多少种不同的取法?

解:取第一个字母的时候会有五种不同的取法,取第二个字母的时候会有四种不同的取法,也就是说一共有$5 \times 4 $ 种可能。


但是每个包含两个字母的组都被计算了两遍,比如说$AB,BA$其实是同一组。但是在计算的时候被计算成了两组。这样重复计算的遍数,取决于要抽取元素的个数。抽取的两个元素照成的结果是重复$2!$遍。可以想象成抽取的元素的个数其实就是盒子的个数,最后给盒子进行排序的情况要从所有结果中排除掉。

$$
\frac{5 \times 4}{ 2 \times 1}
$$

一般来说如果考虑顺序,从$n$ 个元素中随机的找$r$个元素组成一组,一共有$n(n-1)(n-2)(n-3) \cdots (n-r+1)$种不同的方式。而每个含 $r$ 个元素的小组都被重复计算了 $r!$ 次。所以从 $n$ 个元素中找$r$个元素组成不同组的数目为:

$$
\frac
	{n(n-1) \cdots (n-r+1)}
	{ r! } 
= \frac
	{n(n-1) \cdots (n-r+1) ( (n-r) (n-r-1)\cdots 1) }
	{ ( (n-r) (n-r-1)\cdots 1 ) r! }  
= \frac
	{n!}
	{(n-r)!r!} 
$$

对于$r \leqslant n $ ,定义

	$\displaystyle {n \choose r} = \frac{n!}{(n-r)!r!}$

	表示从$n$ 个元素中取$r$个元素的可能组合数。

注意:
$$
{n \choose n} =  {n \choose 0} = \frac{n!}{0! \times n!} = 1
$$


【例】一个有7个人,其中1个坏蛋,6个好人。从7个人中,取出两个人作为一个组合。问取到的都是好人的情况有多少个?取到的坏蛋的情况有多少个?

解:取到的都是好人的情况是,在6个人里面取2个人。

$$
	{ 6 \choose 2} = 15
$$

取到坏人的情况是,在1个坏蛋里面取1个人。在6个好人里面取1个人。

$$
{1 \choose 1}{ 6 \choose 1} = 6
$$

或者是这样想,从7个人里面取2个人也就是一共$\displaystyle {7 \choose 2} = 21 $情况。排除掉所有都取到好人的情况,剩下的就是取到坏人的情况了,一共$21-15=6$种情况。

【例】一共有7个图形,其中2个正方形,5个三角形。从中抽取3个图形做为一组,问所有组合中包括1个正方形的组合有多少?包括2个正方形的组合有多少?3个全部都是三角形的组合有多少?

解:包括1个正方形的组组合,意味着这人组合里面是1个正方形加上2个三角形,也就是说我们要在2个正方形里面先1个正方形,再在5个三角形里面选2个三角形:
$$
	{2 \choose 1}{5 \choose 2} = 20
$$

包括2个正方形的组合,意味着组合里面是2个正方形加上1个三角形,也就是说我们要在2个正方形里面先2个正方形,再在5个三角形里面选1个三角形:

$$
	{2 \choose 2}{5 \choose 1} = 5
$$

3个全部都是三角形的组合,意味着组合里面是3个三角形,也就是说我们要在5个三角形里面选3个三角形:

$$
	{5 \choose 3} = 10
$$

也许你已经注意到,上面三种情况包括了所有组合的可能性。7个图形里面选3个图形的可能性是$\displaystyle {7 \choose 3} = 35 $正好是上面三种情况的总合。

【例】一个团体共有12人,其中5位女士,7位男士,现从中选取2位女士和3位男士组成一个委员会,问有多少种不同的委员会?另外,如果其中2位男土之间有矛盾,并且拒绝一起工作,那又有多少种不同的委员会?

解:可能的组合情况是

$$
{5 \choose 2}{7 \choose 3} = 350
$$

如果其中2位男土之间有矛盾的情况可以想象成上面的例子中,在7个图形里面选3个图形,如果有2个是正方形(不能在一起的同事)在一起的情况:

$$
	{2 \choose 2}{5 \choose 1} = 5
$$

除了这种在一起的情况,剩下的都是不在一起的情况:

$$
	{7 \choose 3}-{2 \choose 2}{5 \choose 1} = 30
$$

选女性委员会成员的方式不变,因此最后的结果是:
$$
{5 \choose 2}  \times 30 = 300
$$

【例】手上的2个球,放到5个不同的盒子中,每个盒子最多放一个球,有多少种不同的放法。

解:转变一下思路,其实这个问题和在5个盒子中选两个盒子的问题是一样的:

$$
{5 \choose 2} = 10
$$

【例】假设在一排n个天线中,有m个是失效的,另n-m个是有效的,并且假设所
有有效的天线之间不可区分,同样,所有失效的天线之间也不可区分,问有多少种线性排列方式,使得任何两个失效的天线都不相邻?

解:假设有效的天线是$\mid$,没有连续的两个失效的天线,也就意味着在失效的天线只能放在每个有效的天线的中间,且只能放一个。下面的 $\sqcup$ 表示的是可以放失效的天线的位置。
$$
	\sqcup \mid \sqcup \mid \sqcup  \mid \sqcup  \cdots \mid \sqcup 
$$

n-m个天线是有效的,那么位置一共是n-m+1个。可以想象成上面的每个位置就像是盒子一样,现在只是把m个天线放到上面的盒子中,可能性为:
$$
{n-m+1 \choose m} 
$$

\begin{tcolorbox}[title = {常用组合恒等式}]{\indent}

$$
{n \choose r} = {n-1 \choose r-1}  + {n-1 \choose r} \qquad 1 \leqslant r \leqslant n
$$

\qquad 证明:设想从n个元素中取r个,一共有${n \choose r}  $ 种取法。从另一个角度来考虑,不妨设这n个元素里有一个特殊的,记为元素1,那么取r个元素就有两种结果:取到元素1或者取不到元素1。

\qquad 取元素1的方法一共有 ${n-1 \choose r-1} $ 种(从n-1个元素里面取r-1个);不取元素1的方法一共有 ${n-1 \choose r } $种(从去掉元素1的剩下n-1个元素中取r个).两者之和就是从n个元素里取r个的方法之和,而从n个元素中取r个共有$ {n \choose r} $种方法,所以式上面的等式成立。
\end{tcolorbox}	

\subsubsection{二项式定理}

二项展开式是依据二项式定理对 $ (a+b)^{n} $进行展开得到的式子,由艾萨克·牛顿于1664-1665年间提出。

\begin{tcolorbox}[title = {二项式定理}]{\indent}

$$
(x+y )^{n} = \sum_{ k=0 }^{ n }  { n \choose k} x^{k}y^{n-k}
$$
\qquad

\end{tcolorbox}	

组合法证明:
$$
(x_{1}+y_{1})(x_{2}+y_{2})\cdots(x_{n}+y_{n})
$$

公式展开后一共包含有$2^{n}$个求和项(每次每项相乘,原本的求和项都会是原来的两倍)。每一项都是n个因子的乘积。而且每一项都包含因子$x_{i}$或者 $y_{i} \quad i =1,2,3\cdots n$,例如:
$$
(x_{1}+y_{1})(x_{2}+y_{2}) = x_{1}x_{2} + y_{1}x_{2} + x_{1}y_{2} + y_{1}y_{2}
$$
$$
(x_{1}+y_{1})(x_{2}+y_{2})(x_{3}+y_{3})  = x_{1}x_{2}x_{3}  + y_{1}x_{2}x_{3}  + x_{1}y_{2}x_{3}  + y_{1}y_{2}x_{3} + x_{1}x_{2}y_{3}  + y_{1}x_{2}y_{3}  + x_{1}y_{2}y_{3}  + y_{1}y_{2}y_{3} 
$$

这$2^{n}$个求和项中,有多少项含有k个$x_{i}$和n-k个$y_{i}$作为因子?

也就是说,在上面($n=3$)的例子中,当($k=1$)时,问有1个含有$x_{i}$与 $2$个含有 $y_{i}$的求和项有几个?分别是:
$$
y_{1}y_{2}x_{3} +  + y_{1}x_{2}y_{3}  + x_{1}y_{2}y_{3}  
$$
相当于是在这3个位置中,选出1个位置,放下$x_{i}$,也就是$\displaystyle {3 \choose 1} $

如果把上面的结论进行一般性的推广的话:含有k个$x_{i}$和n-k个$y_{i}$因子的求和项的个数,对应着从n个元素$x_{1},x_{2},x_{3},x_{4}\cdots x_{n}$中取k个元素的构成的一组取法。因此,一共有$ \displaystyle {n \choose k }$个这样的项。如果每个 $x_{i} = x$, $y_{i} = y$ 这样的话:
$$
(x+y )^{n} = \sum_{ k=0 }^{ n }  { n \choose k} x^{k}y^{n-k}
$$
\par\vspace{\baselineskip}


【例】一个有n个元素的集合共有多少子集?

解: 含有k个元素的子集一共有$ \displaystyle { n \choose k }$个,因此所求答案为:

$$
\sum_{ k=0 }^{ n }  { n \choose k} = 2^{n}
$$

这里的思考过程与上面证明二项系数的求和项的总数的过程是一样的。


\subsubsection{多项式定理}

【例】把n个不同的元素分成r组,每组分别有$n_{1},n_{2},n_{3},n_{4}\cdots n_{i}$个元素,其中$\sum_{i=1}^{n} n_i = n$。问一共有多少种不同的分法?

第一组元素有$\displaystyle { n\choose n_{1} }$种选取方法,选定第一组元素后,只能从剩下的$n-n_{1}$个元素中选。第二组元素,一共有$\displaystyle { n-n_{1} \choose n_{2} }$种取法,接下来第三组有$\displaystyle { n-n_{1}-n_{2} \choose n_{3} }$ 种取法。因此,根据推广的计数基本法则,将n个元素分成r组可能存在:

\begin{align*}  
	&{ n\choose n_{1} }{ n-n_{1} \choose n_{2} }{ n-n_{1}-n_{2} \choose n_{3} }\cdots{ n-n_{1}-n_{2} - \cdots n_{r-1} \choose n_{r} } \\  
	=& \frac{n!}{(n-n_{1})!n_{1}!} \cdot   \frac{(n-n_{1})!}{(n-n_{1}-n_{2})!n_{2}!} \cdots  \frac{(n-n_{1}-n_{2}-\cdots n_{r-1})!}{0! n_{r}!} \\  
	=& \frac{n!}{n_{1}!n_{2}!n_{3}!\cdots n_{r} !}  
\end{align*} 

【例】某个小城的警察局有10名警察,其中5名警察需要在街道巡逻,2名警察需要在局里值班,另外3名留在局里待命.问把10名警察分成这样的3组共有多少种不同分法?

解:一共有10!/(5!×2!×3!)=2520种分法。

【例】把10个孩子平均分成两组进行篮球比赛,一共有多少种分法?

解:这个问题与上例的不同之处在于分成的两组是不用考虑顺序的.也就是说,这里没有A,B两组之分,仅仅分成各自为5人的两组。故所求答案为$\displaystyle \frac{10!/(5! \cdot 5!)}{2!} =126  $



\begin{tcolorbox}[title = {多项式定理}]{\indent}

\textbf{多项式系数}

如果 $ n_{1}+n_{2}+n_{3}+\cdots+n_{r} = n$,则定义$\displaystyle {n \choose n_{1},n_{2},n_{3},\cdots,n_{r}}$为$\displaystyle \frac{n!}{n_{1}!n_{2}!n_{3}!\cdots n_{r} !} $

$\displaystyle {n \choose n_{1},n_{2},n_{3},\cdots,n_{r}}$表示把n个不同的元素分成为$n_{1},n_{2},n_{3},\cdots,n_{r}$一共r个不同组的组合数。

\tcblower

\textbf{多项式定理}
$$
(x_{1}+x_{2}+x_{3}+\cdots+x_{n})^{n}= 
\sum_{(n_{1},n_{2},n_{3}\cdots n_{r}),n_{1}+n_{2}+n_{3}\cdots n_{r} = n}
{n \choose n_{1},n_{2},n_{3},\cdots,n_{r}}
x_{1}^{n_{1}} x_{2}^{n_{2}} x_{3}^{n_{3}}\cdots x_{r}^{n_{r}}
$$

其中:$ n_{i} \geqslant 0, i=1,2,\cdots r$

\end{tcolorbox}	

小结:\\todo 






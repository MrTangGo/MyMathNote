\section{概率论基本概念}
\subsection{随机事件}

\subsection{样本空间与事件}

\subsubsection{样本空间}

某试验所有可能会出现的结果所构成的集合,称之为试验的\textit{ 样本空间(sample space)},并记为$S$。样本空间的元素,即实验的每个结果,称为\textit{ 样本点}。下面是一些样本空间的例子:

(1)试验为考察新生儿的性别,可能结果的集合为$S$=\{ 男,女 \}

(2)试验为同时掷两枚硬币,考察哪一面朝上(记为H)哪一面朝下(记为T)。可能结果的集合为 $S = \{ (H,H),(H,T),(T,H),(T,T)\} $

(3)试验为同时掷两个骰子,考察两个骰子的点数:$S = \{ (1,1),(1,2),(1,3) \cdots  (6,5),(6,6)\}$ 集合中一共36个元素。也可以写成 $ S = \{ (i,j) : i,j = 1,2,3,4,5,6 \}$,i 为第一个骰子的点数,j为第二个骰子的点数。

(4)试验为考察一个电灯的使用寿命(小时), $S = \{ x : 0 \leqslant x < \infty  \}$

\subsubsection{事件}

\textbf{事件的概念}

样本空间的任意一个子集称之为\textit{ 事件(event )},事件就是由试验的某些可能的结果组成的一个集合。如果某个试验的结果(样本点)包含在$E$ 里面,那么就可以说是$E$ 发生了。下面是一些事件的例子:

在(1) 试验中,$E$=\{男\},表示出生的孩子为男生。

在(2)试验中,$E$=\{(H,H),(H,T)\} ,表示事件“第一枚硬币正面朝上”。

在(3)试验中,$E$=\{(1,6),(2,5),(3,4),(4,3),(5,2),(6,1) \},表示事件“两个骰子的点数之合为7”

在(4)试验中,$E=\{x: 0 \leqslant x \leqslant 5 \}$表示事件“电灯的使用寿命不超过5个小时” \par\vspace{\baselineskip}

\textbf{事件的并}

对于任意两个事件:$E$、$F$ ,事件$E\cup F$ 称为事件$E$与事件 $F$的并事件,或和事件。当且仅当$E$、$F$中至少有一个发生时,事件$E\cup F$发生。

在(1)试验中,已知 $E$=\{男\},定义一个新的事件$F$=\{女\} ,$E\cup F$ =\{男,女\},该事件与试验本身的样本空间 $S$ 一致。

在(2)试验中,已知 $E=\{(H,H),(H,T)\}$ ,定义一个新的事件 $F$=\{(T,H)\} ,

$E \cup F= \{(H,H),(H,T),(T,H)\}$,表示“至少有一枚硬币正面朝上”。

设事件$E_{1},E_{2},E_{3} \cdots $是样本空间中的一系列的事件,标记为$ \bigcup_{n=1}^{\infty } E_{n}$表示,至少包含在某一个$E_{n}$里的所有结果所构成的事件。\par\vspace{\baselineskip}


\textbf{事件的交}

对于任意两个事件:$E$、$F$ ,事件$E\cap F$称为$E$ 与 $F$ 的交事件,或积事件。当且仅当$E$、$F$同时发生时,事件$E\cap F$发生。$E\cap F$也写成 $EF$。事件集合$EF$由事件$E$与事件$F$的公共元素组成。

在(2)试验中,设事件 $E=\{(H,H),(H,T),(T,H)\}$ 为“至少一枚硬币朝上”的事件。设事件 $F=\{(H,T),(T,H),(T,T)\}$ 为“至少一枚硬币朝下”的事件。则事件$EF = {(H,T),(T,H)}$为“正好一枚硬币朝上,一枚硬币朝下”。

在(3)试验中,设事件 $E$=\{(1,6),(2,5),(3,4),(4,3),(5,2),(6,1) \},表示事件“两个骰子的点数之合为7”。设事件 $F$=\{(1,5),(2,4),(3,3),(4,2),(5,1) \},表示事件“两个骰子的点数之合为6”。那么事件$EF$不包含任何的试验结果,所以也不可能发生,即$E\cap F = \varnothing $。类似这样的事件可以称之为不可能事件,记为$ \varnothing$,称事件E与F是 \textit{互不相容的(mutually exclusive)}。

设事件$E_{1},E_{2},E_{3}\cdots$是样本空间中的一系列的事件,标记为$ \bigcap_{n=1}^{\infty } E_{n}$表示,包含在所有$E_{n}$里的所有结果所构成的事件。\par\vspace{\baselineskip}


\textbf{补事件}

$S$ 为试验的样本空间,如果$E\cap F = \varnothing $且$E\cup F = S $ 则称事件$E$与事件$F$互为补事件,又称互为对立事件。在每次试验中,事件 $E$与事件 $F$必有一个发生,且仅有一个发生。 $E$的对立事件也可以记为 $\overline{E}$。$\overline{E} = S-E$。$\overline{S}=\varnothing$。\par\vspace{\baselineskip}


\textbf{事件的包含关系}

对于任意的事件$E$和 $F$,如果 $E$ 内的所有结果都在$F$ 中,那么称$E$包含于$F$,记为$E \subset F$。或者说事件$E$发生必然导致事件$F$ 发生。

若$E \subset F$且$F \subset E$,即$E=F$。称事件 $E$与事件 $F$ 相等。\par\vspace{\baselineskip}

\textbf{事件的差}
todo


\textbf{用venn图表示事件之间的关系}
todo

\textbf{事件的运算}





